
\begin{elevator}[Some Applications to 3-connected Graphs]
We prove Tutte's Lemma which states that every 3-connected graph $G$ contains an edge $e$ so that $G/e$ is still 3-connected. While this is an innocuous looking lemma, it allows us to make induction arguments within the subset of graphs that are 3-connected. Using this lemma, we prove that the cycle space of a 3-connected graph has a particularly nice basis. 
\end{elevator}
\label{sec:graph:3connected}

If we remember that  $\mathcal E$ and $\mathcal V$ come from the graph  $G$ they come with additional algebraic structure. The most important structure is the preferred basis (given by the vertices and edges.) While $\mathcal C$ does not come with a preferred basis, it still has a preferred generating set: the cycles. However, there is no reason that the set of cycles should be linearly independent.
\begin{examplefigureenv}[Faces as Basis]{191figures/topgraph_planarcycle.tikz}
	One thing which is interesting to note: when looking at graphs, almost everybody comes up with the same set of cycles for a basis for the cycle space. For instance, in this graph, you would probably say that the three cycles \[\{v_1v_2v_5,v_2v_3v_4, v_1v_5v_3v_4\}\] give a basis for the cycle space. The reason for this is that we are naturally inclined to use the faces from planar drawings as a basis for the cycle space (and it does, indeed, form a basis!) 
\end{examplefigureenv}

We could eliminate a few generators by restricting to a smaller set of cycles.  For example, an \emph{induced cycle} of $G$ is a cycle which cannot be made into 2 smaller cycles with the addition of an edge in $G$. 
\begin{claim}
 The induced cycles of $G$ generate the space $\mathcal C$. 
\end{claim}
This set is still a redundant set of cycles, but without further conditions on the graph there is not a clear candidate of generating cycle.

If a graph $G$ is 3-connected, then there are enough cycles that we can be slightly more picky when choosing a basis for the cycle space. The additional connectivity allows us to pick cycles which do not separate the graph into different components. 
\begin{definition} A cycle $C$ is \emph{non-separating} if $G\setminus C$ is connected. 
\end{definition}
A good example of non-separating cycles comes from a graph drawn in the plane. 
A planar graph is called \emph{polyhedral} if it is 3-connected.
If a planar graph is polyhedral, then every face gives an example of non-separting cycle.
In fact, these are the only non-separating cycles of a planar graph, as every other cycle divides the graph into an interior and exterior region. We will return to this discussion in \sref{sec:planargraphs}.
There is a generalization of this characterization which does not need the planarity requirement. 
\begin{theorem}[Tutte] \label{thm:3connectedgeneration}
 The cycle space of 3-connected graphs is generated by non-separating induced cycles.
\end{theorem}
Before proving this theorem, it is useful to look at an non-example.
 \begin{examplefigureenv}[Tutte Counterexample]{191figures/topgraph_tuttecounter.tikz}First, let's look at a 2-connected graph where this is not true. In this example, the cycle space is 3 dimensional, so one needs to pick 3 independent cycles to get a basis. One can check by hand that every such collection of 3 cycles will necessarily include a generator which is separating.
 \end{examplefigureenv}
% It is not necessarily the case that if you have such a cycle space that your graph is 3-connected. For instance, any tree will trivially satisfy the requirement that its cycle space is generated by non-separating induced cycles. Additionally if a graph has a cycle space generated by non-separating induced cycles, then any subdivion of it does; however, subdivisions are only 2-connected. 

 The proof of this theorem relies crucially on a lemma of Tutte that allows us to make induction-type arguments with 3-connected graphs.
\begin{framedpage}{lemma}{Tutte's Lemma}{
  If $G$  is 3-connected with more than 4 vertices, then there exists an edge $e\in G$ so that $G/e$ is still 3 connected.}
  Suppose for contradiction there is no such edge. Define the function $I: E\times V \to \N$ by
  \[
   I(uv, w)=\text{Size of the smallest component of $G\setminus \{u,v, w\}$}
  \]
	Given our hypotheses we will prove that $I(uv, w)$ has no minimal value which is clearly impossible!

First we show that given our hypothesis, there is a pair $(uv, w)$ so that $G\setminus\{u, v, w\}$ is disconnected. Pick any edge $uv$. By our hypothesis $G/uv$ is   2-connected. Since $G/uv$ is 2-connected but $G$ is not, it must be the case $v_{uv}$ is part of a separating set of $G/uv$. Let $w$ be any vertex such that $(G/uv)\setminus\{v_{uv}, w\}$ is disconnected. Then $u, v, w$ separate $G$.

Now we show that  $I$ has no minimal value.
Pick any $uv, w$ so that $G\setminus\{u, v, w\}$ is disconnected. 
We will show that $I(uv, w)$ is not minimal.
\begin{paragraphfigureenv}{191figures/topgraph_tutteminimum.tikz}
	Let $C$ be the smallest component of $G\setminus\{u, v, w\}$. Because all three vertices are necessary to separate the graph,  $w$ must have a neighbor in the component $C$. Call this vertex $x$. Notice that the neighbors of $x$ are entirely contained in $C$.
	By our assumption, $G/wx$ is 2-connected.
	Therefore, there exists another vertex $y$ so that $G\setminus{w, x, y}$ is disconnected. We will show that this has a connected component which is smaller than $C$.

	Let $D$ be a component of $G\setminus\{w, y, x\}$ which does not contain $uv$. 
	$x$ has a neighbor in $D$, otherwise $G\setminus\{w, y\}$ would be disconnected.
	Therefore $D\cap C$ is nonempty. 
	Furthermore, every vertex of $D$ is contained in $C$, because $D$ is disjoint from the connected components of $G\setminus\{u,v,w\}$ containing $u$ and $v$, and additionally $C$ is a connected subset.
\end{paragraphfigureenv}
  Since $D$ does not contain $x$, we have that $D$ is a proper subset of $C$. Therefore $D$ is smaller size than $C$.  So $I(wx, y)<I(uv, w)$.
  
	\begin{corollary}[Characterization of $\kappa(G)\geq 3$]
		Every 3-connected graph has a $K_4$ minor. 
	\end{corollary}
	The reverse direction holds as well (see \sref{exer:tuttestheorem}.)
\end{framedpage}
\begin{proof}We will use \sref{lemma:3connect} to prove this by induction on the number of edges, and follow the exposition in \cite{diestel2000graph}. Let $e$ be an edge of $G$ so that $G/e$ is 3-connected. 
	By our induction hypothesis, we assume that the theorem holds for $G/e$.
	\begin{paragraphfigureenv}{191figures/topgraph_contractiondecontraction.tikz}
	We prove the theorem by relating the vector spaces  $\mathcal C(G)$ and $\mathcal C (G/e)$. 
Consider the map 
\begin{align*}
	\pi_E:\mathcal E(G)\to \mathcal E(G/e)\\
	\pi_V:\mathcal V(G)\to \mathcal V(G/e)
\end{align*}
constructed in \sref{claim:graphchainmap}. 
We know that there is then a map on the spaces of cycles $\pi: \mathcal C(G)\to \mathcal C(G/e)$. 
In contrast to the example we considered earlier, it is possible that this map has a kernel given by the triangles which are collapsed to an edge under the contraction. 
We call these cycles the \emph{fundamental triangles}, and they span the kernel of the map. 
We can construct chain maps
\begin{align*}
	\i_E:\mathcal E(G/e)\to \mathcal E(G)\\
	\i_V:\mathcal V(G/e)\to \mathcal V(G)
\end{align*}
giving rise to a chain map
\[i:\mathcal C(G/e)\to \mathcal C(G)\]
which is a right inverse to the map $\pi: \mathcal C(G)\to \mathcal C(G/e)$. 
Note that there are many different choices that we could have made in constructing the inverse map $i$. 
\end{paragraphfigureenv}
\begin{claim}
	If $c\in \mathcal C(G)$ is a cycle, and $c\in \ker(\pi)$, then $c$ is a triangle with one edge $e$. The set of such triangles generate $\ker(\pi)$. 
\end{claim}

Let's fix some notation to improve readability of the proof.
 We will call a non-separating induced cycle \emph{basic}.
 Cycles which can be written as the sum of basic cycles will be called \emph{good}. 

	The idea of the proof is to start with a circle $c\in \mathcal C(G)$, obtain a good cycle $\pi(c)\in \mathcal C(G/e)$, then try to lift this back to a new cycle $i\circ \pi(c)$. 
	Because $i$ is only a right inverse there will be a disagreement between $c$ and the lift $i\circ \pi(c)\neq c$.
	We will show that this discrepancy is a good cycle, which completes the proof. 
	\begin{claim}
		Every fundamental triangle is basic.
	\end{claim}
	\begin{proof}
	 Let $C_3$ be a fundamental triangle. If $C_3$ separates $G$, then $C_3/e$ separates $G/e$. But $C_3/e$ only has 2 vertices and  $G/e$ is 3 connected. Therefore $C_3$ is basic. 
	 \begin{claim}
		If $c'\in \mathcal C(G/e)$ is basic, then $i(c')\in \mathcal C(G)$ is good.
	 \end{claim}
	 In the best case, we are in a scenario where 
	 \[(G\setminus i(c'))/e=(G/e)\setminus c',\]
	  which is connected. 
	 It follows that $G\setminus i(C')$ is connected (and we may in fact conclude that $i(C')$ is basic).

	 The more difficult case to handle is when $i(C')$ only passes through one end of the contracted edge $e$.
	 To handle this case, we will need to label the vertices near the contracted edge $e\in G$ containing the possible lifts of $C'$. 
	 \includefigure{191figures/topgraph_contractiondecontraction2.tikz}
	 
In this case there are 2 potential lifts of the cycle $c'$-- the one that goes through $x$, and the one that passes through $y$. We will call these $c_x$ and $c_y$ respectively. Note that if one of $c_x$ or $c_y$ is basic, then the other one is good as they differ by fundamental triangles. 

Suppose for contradiction that neither of these are basic.
Then upon the removal of $c_x$ or $c_y$, the vertices $x$ and $y$ are isolated.
Therefore the only neighbors of $x$ and $y$ are in the set $\{x, y, u, v\}$. The removal of $u$ and $w$ would separate $x, y$ from $G$. But $G$ is supposed to be 3-connected, which contradicts our hypothesis. 
	\end{proof}

We now have all the pieces to complete our proof.
Start with any cycle $c\in \mathcal C(G)$. Because $i$ is an injective map on the cycle space, 
\[c-i\circ \pi (c)\in \ker(\pi),\]
and since $\ker(\pi)$ is generated by fundamental triangles, which are basic, we learn that $c$ is basic whenever $i\circ \pi(c)$  is basic. By the claim, this is a basic cycle, proving the theorem. 
	\end{proof}