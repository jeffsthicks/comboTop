\begin{framedpage}{example}{Some Common Graphs}{It is good to have some examples of graphs in mind before we go around discussing the theory of graphs.}

\begin{smallexamplefigureenv}[Graph:]{191figures/topgraph_graph.tikz}
One way to construct a graph is to build it by hand. 
For example, we can give a graph four vertices and 4 edges by specifying
  \[V=\{1,2,3,4\}\;\;\; E=\{12,23,13,14\}.\]
 While mathematically precise, this presentation is not very intuitive, and so when we want to specify a graph in this text, we will usually just draw a picture of that graph, and assume that it has some explicit (although unwritten) labeling of the vertices and edges.
\end{smallexamplefigureenv}


\begin{smallexamplefigureenv}[Complete Graph:]{191figures/topgraph_completegraph.tikz}
  One especially important family of graphs are the \emph{complete graphs}, which are saturated with edges.
\noindent Let $n\in \N$ be a natural number. The \emph{complete graph on $n$ vertices},  denoted as $K_n$, is the graph with $n$ vertices and an edge between every pair of vertices. 
Since every edge corresponds to a choice of 2 vertices, and in the complete graph we've chosen every pair, it follows that 
\[|E_{K_n}|={n\choose 2}.\]
One reason we call this graph complete is because every graph on fewer than $n$ vertices is a subgraph of $K_n$. 
\end{smallexamplefigureenv}
\nomenclature{$K_n$}{The complete graph with $n$ vertices}


\begin{smallexamplefigureenv}[Octahedron:]{191figures/topgraph_octahedron.tikz}
Graphs naturally arise from other branches of mathematics. 
For example, every polyhedra in $\R^3$ determines a graph by its edges and vertices. In the diagram on the left, we see the graph corresponding to the octahedron.
Later, we will classify the platonic solids by understanding the combinatorics of their corresponding graphs. 
Notice that a polyhedron has more data than just its underlying graph, as it also knows what combinations of edges and vertices make up faces of the polyhedron. 
\end{smallexamplefigureenv}
\end{framedpage}