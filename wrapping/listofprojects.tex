\chapter{List of Project Ideas}
Here is a List of Projects suggested throughout the semester. \\

\noindent \textbf{\hyperref[proj:maxmin]{ Max-Min Principle}}: There is a group of combinatorial results about the duality between connectivity and separability. One form of this that we saw in class was Theorem \ref{thm:menger}, which states that the path connectivity is equivalent to the vertex connectivity. This project could start by looking at a proof of the Max-Flow Min-Cut theorem, which is a powerful generalization of this, and prove that results such as K\"onig's theorem and Dilworths theorem can be rephrased by applying Max-Flow Min-Cut to suitable network flows. \\


\noindent \textbf{\hyperref[proj:spectral]{ Spectral Graph Theory}}: There are a number of linear algebra constructions that we can create given the data of a graph: the adjacency matrix, the degree matrix, Laplacian and differential matrix. Spectral graph theory relates graph properties (or estimates on graphs) to the spectral theory of these matrices. An interesting place to focus would be the isoperimetric number, which numerically characterizes the number of bottlenecks in a graph. 
\begin{theorem}
	Define the \emph{algebraic connectivity $\lambda$} to be second smallest eigenvalue of $L$.
	Define the \emph{diameter} of $G$ to be 
	\[
		D:=\sup_{v, w}\inf_{\text{$P$ from $v$ to $w$}} |P|.
	\]
	Then $\frac{1}{nD}\leq \lambda \leq \kappa(G)$.
\end{theorem}\\

\noindent \textbf{\hyperref[proj:polynomials]{ Graph Polynomials }}: During the class, we'll look at two polynomials related to graph theory, both of them defined by a deletion-contraction relation. This project would look at a characterization of deletion-contraction type invariants via the Tutte polynomial, and elaborate on some of the specializations of the Tutte polynomial. \\

\noindent \textbf{\hyperref[proj:graphhomo]{ Graph Homomorphisms }}\\

\noindent \textbf{\hyperref[proj:matroids]{ Matroids} } Matroids are a far-reaching generalization of graphs. There are many equivalent definitions for a matroid, but in one intuition, one takes a ground set $E$ and specifies which subsets of $E$ are suppose to bye ``independent.'' There are combinatorial restrictions that these independent sets are suppose to satisfy. In graph theory, the ground set are the vertices, and the independent sets are forests of that graph. Many of the tools that we've developed for graph theory naturally generalize to the context of matroids. \\

\noindent \textbf{\hyperref[proj:dualsandhomology]{ Duals and Complexes }} The dual graph is not a concept limited to planar graph, and the existence of a dual linear complex is related to a concept called \emph{Poincar\'e duality} in homology. In this project, one would show that the homology groups for surfaces satisfy Poincar\'e duality using the idea of dual graphs. \\

\noindent \textbf{\hyperref[proj:probmethod]{ Probabilistic Methods in Graphs }}\\


\noindent \textbf{\hyperref[proj:hadwigernelson]{Coloring infinite graphs} } If one tries to color an infinite graph, there are some interesting logical implications! \\

\noindent \textbf{\hyperref[proj:chromatic]{ Zeros of the Chromatic Polynomial} } There are some mysterious relations here! \\

\noindent \textbf{\hyperref[proj:derham]{ De Rahm Cohomology} } Look at the differential geometry parallel to this course. \\

\noindent \textbf{\hyperref[proj:morsetheory]{ Discrete Morse Theory} } Discrete Morse theory is a combinatorial approach to studying the topology by the gradient flow of a function on it. One can capture at first glance properties like the Euler characteristic, but more surprisingly, one can even deduce the homology of the space with enough information about the flow spaces of the Morse function. \\

\noindent \textbf{\hyperref[proj:classification] {Classification of Surfaces.} } There are several ways that one can go about classifying surfaces. One may use the fundamental group, a technique that looks at fundamental polygons, or the uniformization theorem which uses analysis. \\


\noindent \textbf{\hyperref[proj:Abelian] {Abelian Categories.} } In this class we studied homology of spaces by defining them as sequences of vector spaces (or modules.) In fact, the tools that we used can be axiomitized, so that one only needs the rules of an \emph{Abelian category} to make these arguments work. This project would perhaps look at universal properties in Abelian categories to make a few of the arguments we had a bout chain complexes work out. \\


\noindent \textbf{\hyperref[proj:derived] {Triangles.} } Triangles are a mysterious feature of chain complexes. In this project, you would explore the axioms of a \emph{triangulated category}, and show that the zigzag lemma follows from the long exact sequence on cones and the fact that the homotopy category of chain complexes is triangulated. \\


\noindent \textbf{\hyperref[proj:uct] {The Universal Coefficient Theorem} } When we studied homology, we used coefficients coming from $\Z_2$ and $\Z$ to get different invariants about topological spaces. In fact, only coefficients from $\Z$ are required to understand the homology of a space in any other ring $R$, by virtue that every ring $R=R\tensor \Z$. The relationship between $H(X, R)$ and $H(X, \Z)\tensor R$ is captured in the \emph{universal coefficient theorem} which is an algebraic tool for translating coefficients when computing homology.\\

\noindent \textbf{\hyperref[proj:heegaard]{Heegaard Splittings}} A \emph{Heegard Diagram} is a tool which allows us to encode 3-manifolds with planar diagrams. There is a certain set of rules that one is allowed to use to modify these diagrams and arrive at equivalent 3-manifolds. One could look at why these rules necessarily give us a complete combinatorial model for studying 3-manifolds. \\


\noindent \textbf{\hyperref[proj:linkingnumber]{linking Number}} The linking number measures how intertwined two components of a link are. This invariant shows up in many forms; a combinatorial version, a topological definition via the Gauss map, and an integral version which relates to classical electromagnetism. A project looking at the linking number would relate these three viewpoints together. 


\noindent \textbf{\hyperref[proj:knottangle]{Knot Tangle}}


\noindent \textbf{\hyperref[proj:hyperbolicknot]{Hyperbolic Knots}}


\noindent \textbf{\hyperref[proj:groupcoloring]{Group Coloring}}

\noindent \textbf{\hyperref[proj:knotgroup]{Knot Groups}}





