\section{Braids}
\label{sec:knot:braids}
Braid groups describe one particular way that we can tangle stings together. Intuitively, a braid is a tangle of strings that go from top to bottom, that is they proceed in only one direction. The stings of the braid can't intersect, and they can't loop back on themselves. A braid is like a permutation with some additional information encoded about the order that you took the permutations in, and we will formally see later why this is an accurate description of the braid group.
\begin{definition} A collection of functions $f_i:[0,1]\to \R^3$ , $i\in {1,2,\ldots n}$ is called a \textbf{braid on $n$ strings} and is written as $\beta$ if the following properties hold:
\begin{enumerate}
\item $f_i(x) \in \{x\}\times \R^2$
\item $f_i(x)\neq f_j(x)$ for all $i\neq j$
\item $\im f_i$ is a finite collection of line segments
\item $f_i(0)=(i,0,0) $ and $f_i(1)=(\sigma(i),0,1)$ where $\sigma_i\in S_n$ a permutation function.
\end{enumerate}
\end{definition}

\[
BRAIDS?
\]


As one can see, this definition says that a braid is much like a knot, in that it can be represented by a finite collection of line segments, and that it does not intersect itself. How a braid is different than a knot (or a link) is that braids ``travel" in one direction, that is, if we orient the braid by the orientation $[0,1]$, we see that the image of the braid is oriented against a single axis. We will choose this axis to be the $x$ axis, and hence going along the braid means ``going left to right." Like knots, we are interested in braids up to the equivalence of ambient isotopy; when we draw a braid, we really are just picking a representative of the equivalence class of braids on $n$
strings that are isotopic to this one. We will denote the set of all equivalence classes of braids with $n$ strings as $\mathcal{B}_n$. Here are some equivalent braids.
\[
A BRAID
\]


How can we make a group out of braids? The way that designed our braids, we can form a group operation by braid ``stacking'', that is placing one braid on top of each other. 

\begin{definition} Let $\beta_1, \beta_2\in \mathcal{B}_n$, where $\beta_1=\{f_i(t)\}$ and $\beta_2=\{f_i'(t)\}$. Define the new braid $\beta_3=\{g_i\}$ where 
$$g_i= \left\{
\begin{array}{lr} f_i(2t)/2 &:t\in [0,1/2)\\
f_\sigma(i)'(2(t-1/2)/2 +1/2&:t\in [1/2,1]\end{array}\right.$$ 
In abuse of notation, when we write $f_i/2$ we mean a scaling of the $x$ axis by $1/2$ \\
Then we define a group structure on $\mathcal{B}_n$ by $\beta_1 \circ \beta_2 = \beta_3$ Then call $\mathcal{B}_n$ the \textbf{braid group on $n$ strings}. \end{definition}
Of course, we need to check that the axioms for groups are satisfied. The composition of braids is clearly associative: scaling along the $x$ axis is an isotopy of braids, and other than scaling across the $x$ axis, the braids $(\beta_1 \beta_2)\beta_3)$ and $\beta_1(\beta_2\beta_3)$ are the same.
\[
BRAIDS!
\]
We also want to make sure that there exists an identity element and inverse elements. The identity element is easy to come up with: it is simply a set of horizontal strands that have no twisting. 

\[BRAID\]

As for inverses, you can have them too. Simply take the braid $\beta$ you want to inverse, and reflect it along the $yz$ plane, to get a braid $\beta^{-1}$ This should give a braid that, when composed with it's original, is equivalent to the identity. Therefore, we are justified in calling this object the braid group on $n$ strings. Ideally, we would want a group presentation for this object, so we wouldn't have to draw pictures whenever we wanted to talk about a braid. 
It's pretty easy to get generators for the braid group: all braids are generated by the sequential crossing of strands that are adjacent to each other. If we want to designate any braid, we need only the \textbf{braid word} which corresponds to the sequence of strands being crossed. We write the letter $\sigma_i$ to mean the crossing of the $i$ and $i+1$ strands in a right over left fashion. Therefore, the braid given in Figure \ref{braid} is given by the braid word $\sigma_1\sigma_1\sigma_1\sigma_2^{-1}$, or sometimes written $\sigma_1^3\sigma_2^{-1}$. With these braid words, we see that the stacking of braids corresponds to the concatenation of braid words. 
\begin{exercise} Show that the mirror braid of $\beta$ has braid word given by $\beta^{-1}$\end{exercise}

Notice that a braid is not uniquely identified by its braid word: in fact, there are many different braid words that represent the same braid. For instance, the braid $\sigma_3\sigma_1$ is the same as the braid $\sigma_1\sigma_3$. However, the relations giving braids are based mostly on the relations for knots. 
\begin{theorem}
The braid group has a finite presentation as 
\[ \langle \sigma_1, \ldots, \sigma_{n-1} \;|\;\forall |i-j|\geq 2,\;\;  \sigma_{i}\sigma_j=\sigma_j\sigma_i  , \sigma_i \sigma_{i+1} \sigma_i^{-1}=\sigma_{i+1}^{-1}\sigma_i \sigma_{i+1} \rangle. \]
\end{theorem}
Recall that a group presentation is the ``freest group on specified generators that satisfies the given relations. One way of thinking about this is that a group presentation of $G$ specifies a map $F_n\to G$, where $F_n$ is a the free group. The normal closure of the specified generators gives us the kernel of this map. Notice that every finitely generated group has a presentation by the same generators. 
\begin{proof}
The generators represent the elementary crossings of the $i$ and $i+1$ strands. 
These are basically the Reidemeister moves written out as algebraic relations. \\
Notice, that braid isotopy is almost the same as knot isotopy, except that the first Reidemeister move is forbidden! \\
The first type of braid relation corresponds to the ``ambient isotopy.'' It says that if you have two crossings, and those crossings are some distance from each other, then you can commute them past each other because they share no stings in common. This means that crossings ``commute at distance''.\\
The second Reidemeister move is already encoded into the braid group by default: this is just the inverse relation between generators.  \\
The third Reidemeister move, if you write it down, is the relation that we have written above. \\
With these relations, we capture the R-moves, and therefore all the additional structure in the braid group. 
\end{proof}
Notice that if I add in the additional relation that $\sigma_i^2=1$, then I get the permutation group. Interesting!\\
Now, we would like to use braids to study knots. Notice that every braid $\beta$ gives a knot by gluing its ends together in a circle. 
\begin{definition}
Given a braid $\beta$, we say the \emph{braid closure} of $\beta$, $\bar \beta$, is the link created by adding loops from the upper endpoints of the braid to the lower endpoints of the braid in a non-crossing fashion. If $\bar \beta = K$, we say that $\beta$ is a \emph{braid word} for $K$. 
\end{definition}
Given a knot $K$ it is possible to have a lot of different braid words. For instance, both the braid $1$ on one string and the braid $\sigma_1$ on 2 strings are braid words for the unknot. \\
\begin{theorem}
Suppose that $\beta_1, \beta_2$ are two different braid words for $K$. Then they are related by the following \emph{Markov moves}:
\begin{itemize}
\item Concatenation: replace $\beta$ with $\alpha^{-1} \beta \alpha$.
\item Reidemeister 1: Replace $\beta$ on $n$ strings with $\beta\sigma_n$ on $n+1$ strings. 
\end{itemize}
\end{theorem}
\begin{proof}
There are 2 types of  isotopy which are not taken into account by the braid relations: the first one is the isotopy which does not change crossings in the diagram, but possibly moves the braid outside of the ``braided region''. This means that part of the braid moves up over the top of the braided region, and then shows up again on the bottom. You can check that this is the same as concatenating the braid with a word.\\
The other type of isotopy that we have not taken into account is the first Reidemeister move. Suppose we want to do a first Reidemeister move on a little segment. By applying many $R_2$ moves, we can assume that this Reidemeister move occurs on the $n$th string. By applying concatenation, we may assume that it occurs at the end of the braid word. Then the Reidemeister move looks like adding a little loop to the end of the braid. However, adding a little loop to the end of the braid is the same as adding an additional string with a crossing, once we blow up the loop in size a bit. This shows why $R_1$ can be expressed as the replacement of $\beta$ with $\beta \sigma_n$ on $n+1$ strings. 
\end{proof}
So now we know that if we have to braid words, then we can algebraically manipulate one into the other. This is really cool, because on the one hand we've reduced the problem of showing that two knots are the same to an algebraic manipulation. On the other hand, you can prove that it is hard to show that two braid words represent the same knot.\\
But, we do not know if a knot is necessarily representable by a braid. 
\begin{theorem}[Alexander]
For every knot $K$ there is a braid $\beta$ with $\bar \beta = K$. 
\end{theorem}
\begin{proof}
By our assumption about knots, there exists a linearization of the knot image. So, let's pick a linearization of the knot's image so that $K=\bigcup_i L_i$\\
Now, assign an orientation of the knot. Look at a planar projection (knot diagram), and pick a point in the planar projection. If it is the case that each line $L_i$ has an orientation that agrees with the orientation around the point, then we are done! But this is probably not the case. So, for every line segment $L_i$ which disagrees with the orientation around a point $x$, use a bunch of $R_2$ moves to replace it with 2 line segments $L_i'$ and $L_i''$ so that the left endpoint of $L_i'$ is the left endpoint of $L_i$, the right endpoint of $L_i''$ is the right endpoint of $L_i$, and $L_i', L_i''$ have a common endpoint which is on the opposite side of $x$. Now, $L_i'$ and $L_i''$ have an orientation which agrees with $x$. Proceed with this process on every single line segment that has a disagreeable orientation. This provides a knot diagram whose orientation is agreeable with $x$. 
\end{proof}
So, this means that we can take every knot, and find a knot diagram which represents it. Notice that this gives an extremely easy way to produce a Seifert surface, because all of the crossings get unoriented in the same direction. From this piece of data, we should expect that the braid word gives us a purely algebraic way of computing something like the Alexander polynomial for a knot! In fact, it does, but in order to stray away from matrix manipulations for the rest of the course, we will not be going over this one.
